%%% Hlavní soubor. Zde se definují základní parametry a odkazuje se na ostatní části. %%%

%% Verze pro jednostranný tisk:
% Okraje: levý 40mm, pravý 25mm, horní a dolní 25mm
% (ale pozor, LaTeX si sám přidává 1in)
\documentclass[12pt,a4paper]{report}
\setlength\textwidth{145mm}
\setlength\textheight{247mm}
\setlength\oddsidemargin{15mm}
\setlength\evensidemargin{15mm}
\setlength\topmargin{0mm}
\setlength\headsep{0mm}
\setlength\headheight{0mm}
% \openright zařídí, aby následující text začínal na pravé straně knihy
\let\openright=\clearpage

%% Pokud tiskneme oboustranně:
% \documentclass[12pt,a4paper,twoside,openright]{report}
% \setlength\textwidth{145mm}
% \setlength\textheight{247mm}
% \setlength\oddsidemargin{14.2mm}
% \setlength\evensidemargin{0mm}
% \setlength\topmargin{0mm}
% \setlength\headsep{0mm}
% \setlength\headheight{0mm}
% \let\openright=\cleardoublepage

%% Vytváříme PDF/A-2u
\usepackage[a-2u]{pdfx}

%% Přepneme na českou sazbu a fonty Latin Modern
\usepackage[czech]{babel}
\usepackage{lmodern}
\usepackage[T1]{fontenc}
\usepackage{textcomp}

%% Použité kódování znaků: obvykle latin2, cp1250 nebo utf8:
\usepackage[utf8]{inputenc}

%%% Další užitečné balíčky (jsou součástí běžných distribucí LaTeXu)
\usepackage{amsmath}        % rozšíření pro sazbu matematiky
\usepackage{amsfonts}       % matematické fonty
\usepackage{amsthm}         % sazba vět, definic apod.
\usepackage{bbding}         % balíček s nejrůznějšími symboly
			    % (čtverečky, hvězdičky, tužtičky, nůžtičky, ...)
\usepackage{bm}             % tučné symboly (příkaz \bm)
\usepackage{graphicx}       % vkládání obrázků
\usepackage{fancyvrb}       % vylepšené prostředí pro strojové písmo
\usepackage{indentfirst}    % zavede odsazení 1. odstavce kapitoly
\usepackage{natbib}         % zajištuje možnost odkazovat na literaturu
			    % stylem AUTOR (ROK), resp. AUTOR [ČÍSLO]
\usepackage[nottoc]{tocbibind} % zajistí přidání seznamu literatury,
                            % obrázků a tabulek do obsahu
\usepackage{icomma}         % inteligetní čárka v matematickém módu
\usepackage{dcolumn}        % lepší zarovnání sloupců v tabulkách
\usepackage{booktabs}       % lepší vodorovné linky v tabulkách
\usepackage{paralist}       % lepší enumerate a itemize
\usepackage[usenames]{xcolor}  % barevná sazba

%% SPECIMEN
% Takto označená část slouží pro tvorbu vzorového PDF vystaveného na webu.
% Během generování oficiální šablony skriptem ../mkdist se automaticky smaže,
% stejně jako všechna volání maker \X a \XXX.
\def\X#1{\textcolor{red}{[#1]}}
\def\XXX#1{\par\smallskip\noindent \textcolor{red}{[#1]}}
%% NEMICEPS

%%% Údaje o práci

% Název práce v jazyce práce (přesně podle zadání)
\def\NazevPrace{Název práce \X{přesně podle zadání}}

% Název práce v angličtině
\def\NazevPraceEN{Name of thesis \X{přesný překlad do angličtiny}}

% Jméno autora
\def\AutorPrace{Jméno Příjmení}

% Rok odevzdání
\def\RokOdevzdani{ROK}

% Název katedry nebo ústavu, kde byla práce oficiálně zadána
% (dle Organizační struktury MFF UK, případně plný název pracoviště mimo MFF)
\def\Katedra{Název katedry nebo ústavu \X{dle organizační struktury MFF UK}}
\def\KatedraEN{Name of the department}

% Jedná se o katedru (department) nebo o ústav (institute)?
\def\TypPracoviste{Katedra}
\def\TypPracovisteEN{Department}

% Vedoucí práce: Jméno a příjmení s~tituly
\def\Vedouci{Vedoucí práce \X{s~tituly}}

% Pracoviště vedoucího (opět dle Organizační struktury MFF)
\def\KatedraVedouciho{katedra}
\def\KatedraVedoucihoEN{department}

% Studijní program a obor
\def\StudijniProgram{studijní program}
\def\StudijniObor{studijní obor}

% Nepovinné poděkování (vedoucímu práce, konzultantovi, tomu, kdo
% zapůjčil software, literaturu apod.)
\def\Podekovani{%
Poděkování.
}

% Abstrakt (doporučený rozsah cca 80-200 slov; nejedná se o zadání práce)
\def\Abstrakt{%
Abstrakt. \X{doporučeno cca 80--200 slov; nejedná se o~zadání práce}
}
\def\AbstraktEN{%
Abstract.
}

% 3 až 5 klíčových slov (doporučeno), každé uzavřeno ve složených závorkách
\def\KlicovaSlova{%
{klíčová} {slova} \X{obvykle 3 až~5 klíčových slov či frází}
}
\def\KlicovaSlovaEN{%
{key} {words}
}

%% Balíček hyperref, kterým jdou vyrábět klikací odkazy v PDF,
%% ale hlavně ho používáme k uložení metadat do PDF (včetně obsahu).
%% Většinu nastavítek přednastaví balíček pdfx.
\hypersetup{unicode}
\hypersetup{breaklinks=true}

%% Definice různých užitečných maker (viz popis uvnitř souboru)
%%% Tento soubor obsahuje definice různých užitečných maker a prostředí %%%
%%% Další makra připisujte sem, ať nepřekáží v ostatních souborech.     %%%

%%% Drobné úpravy stylu

% Tato makra přesvědčují mírně ošklivým trikem LaTeX, aby hlavičky kapitol
% sázel příčetněji a nevynechával nad nimi spoustu místa. Směle ignorujte.
\makeatletter
\def\@makechapterhead#1{
  {\parindent \z@ \raggedright \normalfont
   \Huge\bfseries \thechapter. #1
   \par\nobreak
   \vskip 20\p@
}}
\def\@makeschapterhead#1{
  {\parindent \z@ \raggedright \normalfont
   \Huge\bfseries #1
   \par\nobreak
   \vskip 20\p@
}}
\makeatother

% Toto makro definuje kapitolu, která není očíslovaná, ale je uvedena v obsahu.
\def\chapwithtoc#1{
\chapter*{#1}
\addcontentsline{toc}{chapter}{#1}
}

% Trochu volnější nastavení dělení slov, než je default.
\lefthyphenmin=2
\righthyphenmin=2

% Zapne černé "slimáky" na koncích řádků, které přetekly, abychom si
% jich lépe všimli.
\overfullrule=1mm

%%% Makra pro definice, věty, tvrzení, příklady, ... (vyžaduje baliček amsthm)

\theoremstyle{plain}
\newtheorem{veta}{Věta}
\newtheorem{lemma}[veta]{Lemma}
\newtheorem{tvrz}[veta]{Tvrzení}

\theoremstyle{plain}
\newtheorem{definice}{Definice}

\theoremstyle{remark}
\newtheorem*{dusl}{Důsledek}
\newtheorem*{pozn}{Poznámka}
\newtheorem*{prikl}{Příklad}

%%% Prostředí pro důkazy

\newenvironment{dukaz}{
  \par\medskip\noindent
  \textit{Důkaz}.
}{
\newline
\rightline{$\qedsymbol$}
}

%%% Prostředí pro sazbu kódu, případně vstupu/výstupu počítačových
%%% programů. (Vyžaduje balíček fancyvrb -- fancy verbatim.)

\DefineVerbatimEnvironment{code}{Verbatim}{fontsize=\small, frame=single}

%%% Prostor reálných, resp. přirozených čísel
\newcommand{\R}{\mathbb{R}}
\newcommand{\N}{\mathbb{N}}

%%% Užitečné operátory pro statistiku a pravděpodobnost
\DeclareMathOperator{\pr}{\textsf{P}}
\DeclareMathOperator{\E}{\textsf{E}\,}
\DeclareMathOperator{\var}{\textrm{var}}
\DeclareMathOperator{\sd}{\textrm{sd}}

%%% Příkaz pro transpozici vektoru/matice
\newcommand{\T}[1]{#1^\top}

%%% Vychytávky pro matematiku
\newcommand{\goto}{\rightarrow}
\newcommand{\gotop}{\stackrel{P}{\longrightarrow}}
\newcommand{\maon}[1]{o(n^{#1})}
\newcommand{\abs}[1]{\left|{#1}\right|}
\newcommand{\dint}{\int_0^\tau\!\!\int_0^\tau}
\newcommand{\isqr}[1]{\frac{1}{\sqrt{#1}}}

%%% Vychytávky pro tabulky
\newcommand{\pulrad}[1]{\raisebox{1.5ex}[0pt]{#1}}
\newcommand{\mc}[1]{\multicolumn{1}{c}{#1}}


%% Titulní strana a různé povinné informační strany
\begin{document}
%%% Titulní strana práce a další povinné informační strany

%%% SPECIMEN
%%% Nápisy na přední straně desek

\pagestyle{empty}
\hypersetup{pageanchor=false}
\XXX{Přední strana pevných desek vazby. Není součástí elektronické verze práce.}
\begin{center}

\large
Univerzita Karlova

\medskip

Matematicko-fyzikální fakulta

\vfill

{\huge\bf BAKALÁŘSKÁ PRÁCE}

\vfill

\hbox to \hsize{\RokOdevzdani\hfil \AutorPrace}

\end{center}

\newpage\openright

%%% NEMICEPS

%%% Titulní strana práce

\pagestyle{empty}
\hypersetup{pageanchor=false}

\begin{center}

\centerline{\mbox{\includegraphics[width=166mm]{../img/logo-cs.pdf}}}

\vspace{-8mm}
\vfill

{\bf\Large BAKALÁŘSKÁ PRÁCE}

\vfill

{\LARGE\AutorPrace}

\vspace{15mm}

{\LARGE\bfseries\NazevPrace}

\vfill

\Katedra

\vfill

\begin{tabular}{rl}

Vedoucí bakalářské práce: & \Vedouci \\
\noalign{\vspace{2mm}}
Studijní program: & \StudijniProgram \\
\noalign{\vspace{2mm}}
Studijní obor: & \StudijniObor \\
\end{tabular}

\vfill

% Zde doplňte rok
Praha \RokOdevzdani

\end{center}

\newpage

%%% NOPHD
%%% Následuje vevázaný list -- kopie podepsaného "Zadání bakalářské práce".
%%% Toto zadání NENÍ součástí elektronické verze práce, nescanovat.
\XXX{Vevázaný list s~kopií podepsaného \uv{Zadání bakalářské práce}. Toto zadání \emph{není} součástí elektronické verze, nescanovat!}
%%% PHDNO

%%% Strana s čestným prohlášením k bakalářské práci

\openright
\hypersetup{pageanchor=true}
\pagestyle{plain}
\pagenumbering{roman}
\vglue 0pt plus 1fill

\noindent
Prohlašuji, že jsem tuto bakalářskou práci vypracoval(a) samostatně a výhradně
s~použitím citovaných pramenů, literatury a dalších odborných zdrojů.
Tato práce nebyla využita k získání jiného nebo stejného titulu.

\medskip\noindent
Beru na~vědomí, že se na moji práci vztahují práva a povinnosti vyplývající
ze zákona č. 121/2000 Sb., autorského zákona v~platném znění, zejména skutečnost,
že Univerzita Karlova má právo na~uzavření licenční smlouvy o~užití této
práce jako školního díla podle §60 odst. 1 autorského zákona.

\vspace{10mm}

\hbox{\hbox to 0.5\hsize{%
V ........ dne ............
\hss}\hbox to 0.5\hsize{%
Podpis autora
\hss}}

\vspace{20mm}
\newpage

%%% Poděkování

\openright

\noindent
\Podekovani

\newpage

%%% Povinná informační strana bakalářské práce

\openright

\vbox to 0.5\vsize{
\setlength\parindent{0mm}
\setlength\parskip{5mm}

Název práce:
\NazevPrace

Autor:
\AutorPrace

\TypPracoviste:
\Katedra

Vedoucí bakalářské práce:
\Vedouci, \KatedraVedouciho

Abstrakt:
\Abstrakt

Klíčová slova:
\KlicovaSlova

\XXX{Informace o~práci se musí objevit i v~metadatech PDF. Přečtěte si v~souboru {\tt README}, jak se to dělá.}
\vss}\nobreak\vbox to 0.49\vsize{
\setlength\parindent{0mm}
\setlength\parskip{5mm}

Title:
\NazevPraceEN

Author:
\AutorPrace

\TypPracovisteEN:
\KatedraEN

Supervisor:
\Vedouci, \KatedraVedoucihoEN

Abstract:
\AbstraktEN

Keywords:
\KlicovaSlovaEN

\vss}

\newpage

\openright
\pagestyle{plain}
\pagenumbering{arabic}
\setcounter{page}{1}


%%% Strana s automaticky generovaným obsahem bakalářské práce

\tableofcontents

%%% Jednotlivé kapitoly práce jsou pro přehlednost uloženy v samostatných souborech
\include{uvod}
\include{kap01}
\include{kap02}
\include{kap03}
%%% Fiktivní kapitola s instrukcemi k PDF/A

\chapter{Formát PDF/A}

Opatření rektora č. 23/2016 určuje, že elektronická podoba závěrečných
prací musí být odevzdávána ve formátu PDF/A úrovně 1a nebo 2u. To jsou
profily formátu PDF určující, jaké vlastnosti PDF je povoleno používat,
aby byly dokumenty vhodné k~dlouhodobé archivaci a dalšímu automatickému
zpracování. Dále se budeme zabývat úrovní 2u, kterou sázíme \TeX{}em.

Mezi nejdůležitější požadavky PDF/A-2u patří:

\begin{itemize}

\item Všechny fonty musí být zabudovány uvnitř dokumentu. Nejsou přípustné
odkazy na externí fonty (ani na \uv{systémové}, jako je Helvetica nebo Times).

\item Fonty musí obsahovat tabulku ToUnicode, která definuje převod z~kódování
znaků použitého uvnitř fontu to Unicode. Díky tomu je možné z~dokumentu
spolehlivě extrahovat text.

\item Dokument musí obsahovat metadata ve formátu XMP a je-li barevný,
pak také formální specifikaci barevného prostoru.

\end{itemize}

Tato šablona používá balíček {\tt pdfx,} který umí \LaTeX{} nastavit tak,
aby požadavky PDF/A splňoval. Metadata v~XMP se generují automaticky podle
informací v~souboru {\tt prace.xmpdata} (na vygenerovaný soubor se můžete
podívat v~{\tt pdfa.xmpi}).

Validitu PDF/A můžete zkontrolovat pomocí nástroje VeraPDF, který je
k~dispozici na \url{http://verapdf.org/}.

Pokud soubor nebude validní, mezi obvyklé příčiny patří používání méně
obvyklých fontů (které se vkládají pouze v~bitmapové podobě a/nebo bez
unicodových tabulek) a vkládání obrázků v~PDF, které samy o~sobě standard
PDF/A nesplňují.

Je pravděpodobné, že se to týká obrázků vytvářených mnoha různými programy.
V~takovém případě se můžete pokusit obrázek do zkonvertovat do PDF/A pomocí
GhostScriptu, například takto:

\begin{verbatim}
        gs -q -dNOPAUSE -dBATCH
           -sDEVICE=pdfwrite -dPDFSETTINGS=prepress
           -sOutputFile=vystup.pdf vstup.pdf
\end{verbatim}

Jelikož PDF/A je v~\TeX{}ovem světě (a~nejen tam) novinkou, budeme rádi
za vaše zkušenosti.


\include{zaver}

%%% Seznam použité literatury
\include{literatura}

%%% Obrázky v bakalářské práci
%%% (pokud jich je malé množství, obvykle není třeba seznam uvádět)
\listoffigures

%%% Tabulky v bakalářské práci (opět nemusí být nutné uvádět)
%%% U matematických prací může být lepší přemístit seznam tabulek na začátek práce.
\listoftables
\XXX{U~matematických prací může být lepší přemístit seznam tabulek na začátek práce.}

%%% Použité zkratky v bakalářské práci (opět nemusí být nutné uvádět)
%%% U matematických prací může být lepší přemístit seznam zkratek na začátek práce.
\chapwithtoc{Seznam použitých zkratek}
\XXX{U~matematických prací může být lepší přemístit seznam zkratek na začátek práce.}

%%% Přílohy k bakalářské práci, existují-li. Každá příloha musí být alespoň jednou
%%% odkazována z vlastního textu práce. Přílohy se číslují.
%%%
%%% Do tištěné verze se spíše hodí přílohy, které lze číst a prohlížet (dodatečné
%%% tabulky a grafy, různé textové doplňky, ukázky výstupů z počítačových programů,
%%% apod.). Do elektronické verze se hodí přílohy, které budou spíše používány
%%% v elektronické podobě než čteny (zdrojové kódy programů, datové soubory,
%%% interaktivní grafy apod.). Elektronické přílohy se nahrávají do SISu a lze
%%% je také do práce vložit na CD/DVD. Povolené formáty souborů specifikuje
%%% opatření rektora č. 13/2017.
\appendix
\chapter{Přílohy}
\XXX{Přílohy k bakalářské práci, existují-li. Každá příloha musí být alespoň jednou odkazována z vlastního textu práce. Přílohy se číslují.}
\XXX{Do tištěné verze se spíše hodí přílohy, které lze číst a prohlížet (dodatečné tabulky a grafy, různé textové doplňky, ukázky výstupů z počítačových programů, apod.). Do elektronické verze se hodí přílohy, které budou spíše používány v~elektronické podobě než čteny (zdrojové kódy programů, datové soubory, interaktivní grafy apod.). Elektronické přílohy se nahrávají do SISu a lze je také do práce vložit na CD/DVD. Povolené formáty souborů specifikuje opatření rektora č.~13/2017.}

\section{První příloha}

\openright
\end{document}
