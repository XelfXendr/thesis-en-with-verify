%%% The main file. It contains definitions of basic parameters and includes all other parts.

% Meta-data of your thesis (please edit)
\input metadata.tex

% Generate metadata in XMP format for use by the pdfx package
\input xmp.tex

%% Settings for single-side (simplex) printing
% Margins: left 40mm, right 25mm, top and bottom 25mm
% (but beware, LaTeX adds 1in=25.4mm implicitly)
\documentclass[12pt,a4paper]{report}
\setlength\textwidth{145mm}
\setlength\textheight{247mm}
\setlength\oddsidemargin{14.6mm}
\setlength\evensidemargin{14.6mm}
\setlength\topmargin{0mm}
\setlength\headsep{0mm}
\setlength\headheight{0mm}
% \openright makes the following text appear on a right-hand page
\let\openright=\clearpage

%% Settings for two-sided (duplex) printing
% \documentclass[12pt,a4paper,twoside,openright]{report}
% \setlength\textwidth{145mm}
% \setlength\textheight{247mm}
% \setlength\oddsidemargin{14.6mm}
% \setlength\evensidemargin{0mm}
% \setlength\topmargin{0mm}
% \setlength\headsep{0mm}
% \setlength\headheight{0mm}
% \let\openright=\cleardoublepage

%% If the thesis has no printed version, symmetric margins look better
% \documentclass[12pt,a4paper]{report}
% \setlength\textwidth{145mm}
% \setlength\textheight{247mm}
% \setlength\oddsidemargin{7.1mm}
% \setlength\evensidemargin{7.1mm}
% \setlength\topmargin{0mm}
% \setlength\headsep{0mm}
% \setlength\headheight{0mm}
% \let\openright=\clearpage

%% Generate PDF/A-2u
\usepackage[a-2u]{pdfx}

%% Prefer Latin Modern fonts
\usepackage{lmodern}
% If we are not using LuaTeX, we need to set up character encoding:
\usepackage{iftex}
\ifpdftex
\usepackage[utf8]{inputenc}
\usepackage[T1]{fontenc}
\usepackage{textcomp}
\fi

%% Further useful packages (included in most LaTeX distributions)
\usepackage{amsmath}        % extensions for typesetting of math
\usepackage{amsfonts}       % math fonts
\usepackage{amsthm}         % theorems, definitions, etc.
\usepackage{bm}             % boldface symbols (\bm)
\usepackage{booktabs}       % improved horizontal lines in tables
\usepackage{caption}        % custom captions of floating objects
\usepackage{dcolumn}        % improved alignment of table columns
\usepackage{floatrow}       % custom float environments
\usepackage{graphicx}       % embedding of pictures
\usepackage{indentfirst}    % indent the first paragraph of a chapter
\usepackage[nopatch=item]{microtype}   % micro-typographic refinement
\usepackage{paralist}       % improved enumerate and itemize
\usepackage[nottoc]{tocbibind} % makes sure that bibliography and the lists
			    % of figures/tables are included in the table
			    % of contents
\usepackage{xcolor}         % typesetting in color

% The hyperref package for clickable links in PDF and also for storing
% metadata to PDF (including the table of contents).
% Most settings are pre-set by the pdfx package.
\hypersetup{unicode}
\hypersetup{breaklinks=true}

% Packages for computer science theses
\usepackage{algpseudocode}  % part of algorithmicx package
\usepackage{algorithm}
\usepackage{fancyvrb}       % improved verbatim environment
\usepackage{listings}       % pretty-printer of source code

% You might want to use cleveref for references
% \usepackage{cleveref}

% Set up formatting of bibliography (references to literature)
% Details can be adjusted in macros.tex.
%
% BEWARE: Different fields of research and different university departments
% have their own customs regarding bibliography. Consult the bibliography
% format with your supervisor.
%
% The basic format according to the ISO 690 standard with numbered references
\usepackage[natbib,style=iso-numeric,sorting=none]{biblatex}
% ISO 690 with alphanumeric references (abbreviations of authors' names)
%\usepackage[natbib,style=iso-alphabetic]{biblatex}
% ISO 690 with references Author (year)
%\usepackage[natbib,style=iso-authoryear]{biblatex}
%
% Some fields of research prefer a simple format with numbered references
% (sorting=none tells that bibliography should be listed in citation order)
%\usepackage[natbib,style=numeric,sorting=none]{biblatex}
% Numbered references, but [1,2,3,4,5] is compressed to [1-5]
%\usepackage[natbib,style=numeric-comp,sorting=none]{biblatex}
% A simple format with alphanumeric references:
%\usepackage[natbib,style=alphabetic]{biblatex}

% Load the file with bibliography entries
\addbibresource{bibliography.bib}

% Our definitions of macros (see description inside)
\input macros.tex

%%% Title page and various mandatory informational pages
\begin{document}
%%% Title page of the thesis and other mandatory pages

%%% SPECIMEN
%%% Inscriptions at the opening page of the hard cover

\pagestyle{empty}
\hypersetup{pageanchor=false}
\XXX{Opening page of the hard cover. Not a part of the electronic version.}
\begin{center}

\large
Charles University

\medskip

Faculty of Mathematics and Physics

\vfill

{\huge\bf BACHELOR THESIS}

\vfill

\hbox to \hsize{\YearSubmitted\hfil \ThesisAuthor}

\end{center}

\newpage\openright

%%% NEMICEPS

%%% Title page of the thesis

\pagestyle{empty}
\hypersetup{pageanchor=false}
\begin{center}

\centerline{\mbox{\includegraphics[width=166mm]{../img/logo-en.pdf}}}

\vspace{-8mm}
\vfill

{\bf\Large BACHELOR THESIS}

\vfill

{\LARGE\ThesisAuthor}

\vspace{15mm}

{\LARGE\bfseries\ThesisTitle}

\vfill

\Department

\vfill

{
\centerline{\vbox{\halign{\hbox to 0.45\hsize{\hfil #}&\hskip 0.5em\parbox[t]{0.45\hsize}{\raggedright #}\cr
Supervisor of the bachelor thesis:&\Supervisor \cr
\noalign{\vspace{2mm}}
Study programme:&\StudyProgramme \cr
\noalign{\vspace{2mm}}
Study branch:&\StudyBranch \cr
}}}}

\vfill

% Zde doplňte rok
Prague \YearSubmitted

\end{center}

\newpage

%%% NOPHD
%%% Here should be a bound sheet included -- a signed copy of the "bachelor
%%% thesis assignment". This assignment is NOT a part of the electronic
%%% version of the thesis. DO NOT SCAN.
\XXX{Bound into the introductory part must be the form with signed approval of the thesis topic (a photocopy suffices). This is not a~part of the electronic version of the thesis, do not scan!}
%%% PHDNO

%%% A page with a solemn declaration to the bachelor thesis

\openright
\hypersetup{pageanchor=true}
\pagestyle{plain}
\pagenumbering{roman}
\vglue 0pt plus 1fill

\noindent
I declare that I carried out this bachelor thesis independently, and only with the cited
sources, literature and other professional sources. It has not been used to obtain another
or the same degree.

\medskip\noindent
I understand that my work relates to the rights and obligations under the Act No.~121/2000 Sb.,
the Copyright Act, as amended, in particular the fact that the Charles
University has the right to conclude a license agreement on the use of this
work as a school work pursuant to Section 60 subsection 1 of the Copyright~Act.

\vspace{10mm}

\hbox{\hbox to 0.5\hsize{%
In \hbox to 6em{\dotfill} date \hbox to 6em{\dotfill}
\hss}\hbox to 0.5\hsize{\dotfill\quad}}
\smallskip
\hbox{\hbox to 0.5\hsize{}\hbox to 0.5\hsize{\hfil Author's signature\hfil}}

\vspace{20mm}
\newpage

%%% Dedication

\openright

\noindent
\Dedication

\newpage

%%% Mandatory information page of the thesis

\openright

\vbox to 0.5\vsize{
\setlength\parindent{0mm}
\setlength\parskip{5mm}

Title:
\ThesisTitle

Author:
\ThesisAuthor

\DeptType:
\Department

Supervisor:
\Supervisor, \SupervisorsDepartment

Abstract:
\Abstract

Keywords:
\Keywords

\XXX{This information must be stored as PDF meta-data, too. Please refer to the {\tt README} file.}
\vss}

\newpage

\openright
\pagestyle{plain}
\pagenumbering{arabic}
\setcounter{page}{1}


%%% A page with automatically generated table of contents of the thesis

\tableofcontents

%%% Each chapter is kept in a separate file
\chapter*{Introduction}
\addcontentsline{toc}{chapter}{Introduction}


\chapter{Title of the first chapter}

An~example citation: \cite{Andel07}

\section{Title of the first subchapter of the first chapter}

\section{Title of the second subchapter of the first chapter}

\chapter{Title of the second chapter}

\section{Title of the first subchapter of the second chapter}

\section{Title of the second subchapter of the second chapter}


\chapter*{Conclusion}
\addcontentsline{toc}{chapter}{Conclusion}


%%% Bibliography
%%% Bibliography (literature used as a source)
%%%
%%% We employ biblatex to construct the bibliography. It processes
%%% citations in the text (e.g., the \cite{...} macro) and looks up
%%% relevant entries in the bibliography.bib file.
%%%
%%% See also biblatex settings in thesis.tex.

%%% Generate the bibliography. Beware that if you cited no works,
%%% the empty list will be omitted completely.

% We let bibliography items stick out of the right margin a little
\def\bibfont{\hfuzz=2pt}

\printbibliography[heading=bibintoc]

%%% If case you prefer to write the bibliography manually (without biblatex),
%%% you can use the following. Please follow the ISO 690 standard and
%%% citation conventions of your field of research.

% \begin{thebibliography}{99}
%
% \bibitem{lamport94}
%   {\sc Lamport,} Leslie.
%   \emph{\LaTeX: A Document Preparation System}.
%   2nd edition.
%   Massachusetts: Addison Wesley, 1994.
%   ISBN 0-201-52983-1.
%
% \end{thebibliography}


%%% Figures used in the thesis (consider if this is needed)
\listoffigures

%%% Tables used in the thesis (consider if this is needed)
%%% In mathematical theses, it could be better to move the list of tables to the beginning of the thesis.
\listoftables

%%% Abbreviations used in the thesis, if any, including their explanation
%%% In mathematical theses, it could be better to move the list of abbreviations to the beginning of the thesis.
\chapwithtoc{List of Abbreviations}

%%% Doctoral theses must contain a list of author's publications
\ifx\ThesisType\TypePhD
\chapwithtoc{List of Publications}
\fi

%%% Attachments to the thesis, if any. Each attachment must be referred to
%%% at least once from the text of the thesis. Attachments are numbered.
%%%
%%% The printed version should preferably contain attachments, which can be
%%% read (additional tables and charts, supplementary text, examples of
%%% program output, etc.). The electronic version is more suited for attachments
%%% which will likely be used in an electronic form rather than read (program
%%% source code, data files, interactive charts, etc.). Electronic attachments
%%% should be uploaded to SIS. Allowed file formats are specified in provision
%%% of the rector no. 72/2017. Exceptions can be approved by faculty's coordinator.
\appendix
\chapter{Attachments}

\section{First Attachment}

\end{document}
